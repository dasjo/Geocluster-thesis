
%
% use cases
%

\chapter{Use cases}

\section{Demo Use Case Features}

A set of demonstration use cases has been created in order to test and evaluate the Geocluster implementation described in chapter \ref{chapter:architecture-implementation}. The set consists of one non-clustering map and 3 maps based on the different clustering algorithms. The demo use cases were configured using various Drupal modules and exported into code using the Features module\footnote{\url{http://drupal.org/project/features}}.

\begin{itemize}

\item \textbf{Geocluster Demo} show cases maps based on the two clustering algorithms provided by Geocluster module: PHP-based clustering and MySQL-based clustering and an additional map that doesn't use clustering at all. The article content type of a standard Drupal installation is extended by a Geofield-based place field for storing locations. For each map, a separate View is configured to provide a GeoJSON feed. A Leaflet map is then added on top of the feed by using the Leaflet GeoJSON module.

\item \textbf{Geocluster Demo Solr} adds a show case of the Solr-based clustering algorithm. It provides a setup based on Views GeoJSON and Leaflet GeoJSON similar to the Geocluster Demo feature. In addition, a Search API Server and Index configuration is added for indexing and querying the data using Apache Solr.  

\item \textbf{Geocluster Demo Content} is a sub-module that automatically imports a set of demo content for testing the Geocluster Demo and Geocluster Demo Solr features.

\end{itemize}

\section{GeoRecruiter}

The practical use case for server-side geo clustering should add spatial search capabili- ties to the Recruiter job board solution.
Recruiter is a Drupal distribution for building Drupal based e-recruitment platforms. Users can register either as recruiter and post job classifieds or they can register as applicants and fill out their resume. A faceted search helps users to find jobs and possible job candidates.1
Adding server-side geo clustering capabilities would allow to visualize several thou- sands of available jobs on an interactive map for large-scale e-recruitment websites. The server-side clustering solution should be designed for the possibility to be added to geospatial searches realized in combination with the Recruiter distribution. This influ- ences the integration and extensibility requirements, stated in the previous chapter.

As a practical use case, integrating the server-side geoclustering capabilities of Geocluster into the Recruiter distribution was planned. While the technical groundwork has been made, the actual use case hasn't been implemented, yet.

GeoRecruiter will allow to visualize several thousands of available jobs on an interactive map for large-scale e-recruitment websites. The Geocluster Solr module is designed to provide the clustering capabilities needed by GeoRecruiter. The Solr-based aggregation integrates well with the architecture of the Recruiter distribution and is designed for scalability up to 1,000,000 indexed jobs as evaluated in chapter \ref{chapter:performance}.

Besides the clustering functionality, GeoRecruiter will support location-based search. This allows the user to search for jobs within the surroundings of a desired region by applying a proximity filter. The Search API Location module is currently being refactored\footnote{\url{http://drupal.org/node/1798168}} in order to provide a solid foundation for such spatial queries using Solr and the Search API module suite.