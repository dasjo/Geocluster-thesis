
%
% abstract
%


\chapter*{Abstract}

TODO2

\newpage

% and in German
\chapter*{Kurzfassung}

\hyphenation{Ab-sol-ven-ten}
\hyphenation{vor-zu-schla-gen}

TODO Systeme unterstützen Benutzer die richtigen Artikel aus einer großen Auswahlmöglichkeit zu finden. Absolventen.at, eine österreichische Jobplattform für Absolventen, verwendet ein solches System, um Bewerbern passende Jobs vorzuschlagen. Bis jetzt wurde nur der Lebenslauf als Bestandteil des Benutzerprofils, welches mit den verfügbaren Jobs verglichen wird, betrachtet. Jedoch füllen nur rund die Hälfte aller registrierten Benutzer einen solchen Lebenslauf aus, für die andere Hälfte können keine personalisierten Empfehlungen generiert werden. Um dies zu verbessern, wurde das System mit implizitem Relevanz Feedback erweitert und die Auswirkungen dieses Ansatzes wurden in dieser Diplomarbeit untersucht. Implizites Feedback kann in einer unauffälligen Weise aufgezeichnet werden und ermöglicht dem System Benutzerpräferenzen abzuleiten. Vier unterschiedliche Benutzeraktionen konnten auf Absolventen.at für implizites Feedback identifiziert werden, darunter Lesen einer Stellenbeschreibung, Hinzufügen von Lesezeichen, Schreiben von Bewerbungen und Suchen nach Jobs. Jede dieser Aktionen liefert unterschiedliche Evidenzen für Interesse. So ist eine Bewerbung ein zuverlässigerer Indikator für Interesse als nur das Lesen einer Stellenbeschreibung, was durch unterschiedliche Gewichtungsparameter berücksichtigt wird. Zusätzlich werden graduelle Vergessensfaktoren verwendet, um das Profil über die Zeit zu adaptieren. All diese Informationen werden im hybriden Benutzerprofil miteinbezogen, welches als hyperdimensionaler Vektor repräsentiert und mit Hilfe einer Linearkombination aus dem Lebenslauf und den bevorzugten Jobs berechnet wird.  Für die Evaluierung des neuen Ansatzes wurden die bevorzugten Jobs von 46 Jobsuchenden mit dem Empfehlungen verglichen. Die Ergebnisse zeigen, dass implizites Feedback hilfreich ist, um sowohl die Benutzerreichweite als auch die Treffergenauigkeit der Empfehlungen zu erhöhen.
