
%
% realization - analysis
%

\section{Analysis}
\label{chapter:analysis}

Two objectives stated in chapter \ref{chapter:objectives} are 1) \textit{performance} and 3) \textit{integration and extensibility}. Together, they define the target of creating an integrated solution for server-side clustering in Drupal with a clear focus on enhancing the performance of data-intense maps.


\subsection{Algorithm considerations}

By the definition of clustering foundations explained in \ref{chapter:foundations-clustering}, a number of factors need to be considered when designing the clustering algorithm. For the clustering task, a \textit{pattern representation} method, a \textit{proximity measure} and the \textit{clustering algorithm} itself need to be defined. In addition, the \textit{cluster type} and the right choice of \textit{clustering techniques} have to be considered. The following enumeration discusses the considerations for the named factors. 

\begin{itemize}

\item \textbf{Pattern representation}: A number of spatial data types as points, lines or rectangles exist (see chapter \ref{chapter:spatial-data}). For practical and simplicity reasons, only points as the simplest spatial data type need to considered for the pattern representation of the server-side clustering task. The exact selection of the pattern representation depends on the Drupal integration consideration, outlined in the following chapter \ref{chapter:analysis-drupal}.

\item \textbf{Proximity measure}: Amongst the options for proximity measures (see chapter \ref{chapter:proximity}), the Euclidean distance is the obvious choice for the intended clustering task. What needs to be considered in this case though, are the implications of a map projection being used to represent the spherical geoid on a planar map on the computer screen (see \ref{chapter:projections}).

\item \textbf{Cluster type}: Different means for the definition of cluster types have been explained in chapter \ref{chapter:cluster-types}. The attributes of \textit{well-separated}, \textit{prototype-based} and \textit{density-based} clusters seem logical for clustering points on a map. Most importantly, clusters should be defined by the centroid of all clustered items as their prototype. The attributes of the two other cluster types only apply to a certain extend. Well-separated ensures that clusters don't overlap which enhances readability of the map. Density-based clusters account for a visual grouping of items in crowded regions. For simplicity and readability, prototype-based clusters represented as single map markers are preferred over potentially polymorph well-separated or density-based clusters.

\item \textbf{Clustering algorithm}: In order to support dynamic queries, the clustering task needs to be performed on-the-fly. On the other hand, the clustering should perform efficiently. Chapter \ref{chapter:clustering-grid} explains how the grid-based STING algorithm precalculates clusters in order to achieve a constant time complexity for the actual retrieval of cluster items at query-time. The intended design and implementation of the server-side clustering algorithm needs find a good balance between performance while still guaranteeing an on-the-fly clustering of dynamically retrieved data sets.

\end{itemize}


\subsection{Drupal integration considerations}
\label{chapter:analysis-drupal}

Drupal already provides a variety of tools in order to create interactive maps. The following chapter analyses how a server-side clustering implementation could integrate with existing Drupal mapping tools that have been explained in chapter \ref{chapter:drupal-mapping}.

\begin{itemize}

\item \textbf{Configuration}: Most Drupal modules provide a user interface that helps configure settings. Similarly, the server-side clustering implementation should be configurable using a user interface in order to setup and parametrize the clustering process. Drupal 7 includes a versatile \textit{Form API\footnote{\url{http://api.drupal.org/api/drupal/developer!topics!forms_api_reference.html/7}}} that helps build such forms and potentially should be used to achieve this requirement.

\item \textbf{Storage}: The de-facto standard for storing geospatial data in Drupal 7 is the Geofield module. By default, it doesn't provide a \textit{spatial index} but stores spatial data in the database field table as separate columns for latitude, longitude and other related spatial information as the bounding box. Given the popularity of the Geofield module, it should be considered as the primary source for spatial data to be processed within the server-side clustering implementation.

The Recruiter distribution uses the Search API module suite for performant queries using the Apache Solr search platform. In order to integrate the server-side clustering solution with Recruiter, a possibility for indexing the spatial data using in Solr using the Search API module has to be found.

\item \textbf{Querying}: Drupal integration on the query level primarily needs to happen in combination with the Views and Views GeoJSON modules. The clustering task therefore needs to be integrated into the process of how the Views module queries data and processes its results. The Views module provides an extensive API\footnote{\url{http://api.drupal.org/api/views}} that allows to extend its functionality.

Two main challenges of integrating a server-side clustering solution with Views have been identified: 1) \textit{allow to inject a custom aggregation implementation}\footnote{\url{http://drupal.org/node/1791796}} and 2) \textit{dealing with geospatially clustered data}\footnote{\url{http://drupal.org/node/1824954}}. The first challenge deals with finding a clean way to integrate a clustering solution into the processing queue of the Views module. The second subsequently deals with challenges that arise when processing and visualizing the clustered data afterwards. As the clustering process changes the data being processed, the implementation needs to take care of involved APIs that work with the changed data. 

Similarly to the previously discussed storage aspect of the Drupal integration, the querying component to account for clustering data in combination with Solr and the Search API.

\item \textbf{Visualization}: Once clustered, the data needs to be visualized on the client. Requests based on the Bounding Box strategy in combination with a JavaScript mapping library will supply the clustered data for the client. The clustered data then needs to be visualized appropriately.

\end{itemize}


Based on the previous insights regarding the algorithm and Drupal integration considerations, a concrete clustering algorithm needed to be found. As explained, multiple data storage backends including the default MySQL-based Geofield storage and Apache Solr as indexing server are considered which means that the clustering algorithm should not rely on a particular database. The Geohash encoding algorithm for spatial coordinates into string identifiers has been explained in chapter \ref{chapter:geohash}. Its hierarchical spatial structure can be leveraged as a spatial indexing mechanism that is abstracted from the concrete database implementation. In the following, the concrete algorithm, architecture and implementation will be explained.









