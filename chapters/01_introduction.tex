
%
% intro
%

\chapter{Introduction}


\section{Motivation}

Digital mapping applications on the Internet are strongly emerging. Big players like Google Maps\footnote{\url{https://maps.google.at}} and OpenStreetMap\footnote{\url{http://www.openstreetmap.org/ }} provide online maps, that users can view and interact with. 

Maps allow telling stories and communicating data in a visual way. Using open source tools such as TileMill\footnote{\url{http://mapbox.com/tilemill}} and online services like CloudMade\footnote{\url{http://cloudmade.com/ }}, more and more people are able to create their own custom maps. The Free and Open Source content management systems and framework Drupal\footnote{\url{http://drupal.org}} provides tools to add, edit and visualize geographic data on maps. This allows to integrate interactive map applications into web sites.

Visualizing thousands of points on a single map are both challenges to human and digital computability. Obviously, when telling a story, information needs to be told in a compact way as the human brain can only process a limited amount of data at the same time. Similarly, large amounts of data involve a higher burden on the computer components that participate in the web mapping process.

Clustering\footnote{\url{http://en.wikipedia.org/wiki/Cluster_analysis}} is a technique for grouping objects with similarities. Various Javascript libraries like Leaflet.markercluster\footnote{\url{https://github.com/Leaflet/Leaflet.markercluster}} exist for clustering points on maps on the client-side. This enhances performance and readability of data-heavy map applications. But still, all data needs to be transferred to the client and processed on a potentially slower end user device. By clustering data on the server-side, the load is shifted from the client to the server which allows displaying larger amounts of data in a performant way. Professional services like maptimize\footnote{\url{http://www.maptimize.com}} provide such a functionality, while in the open source space little libraries and frameworks exist for server-side clustering of geospatial data.

In order to create interactive maps based on large data sets, this thesis evaluates and implements a performance-optimized server-side clustering algorithm for Drupal.


\section{Outline of the thesis}

Chapter \ref{chapter:foundations} introduces the technical foundations for this thesis. It provides an overview by explaining cluster theory and comparing four main clustering algorithms. Further, a discussion on spatial data computation introduces space order and space decomposition methods, Quadtrees and the Geohash algorithm. Finally, an overview of foundational concepts in web mapping as coordinate systems, map projections and spatial data types is provided. 

Chapter \ref{chapter:state} discusses the state of the art of related technologies for the thesis. An explanation of a modern web mapping stack is given as well as the basics of Drupal \& mapping technologies. Further, the state of the art for client-side and server-side clustering technologies in web mapping is analyzed.

Chapter \ref{chapter:objectives} states the objectives for the thesis. It defines the goals to be accomplished as a result of the implementation part of the thesis.

Chapter \ref{chapter:realization} describes the realization of the server-side clustering algorithm. First, an analysis based on the objectives stated in the previous chapter is given. Subsequently, the Geohash-based clustering algorithm is defined. Finally, the architecture and implementation of the algorithm for Drupal is explained in detail.

Chapter \ref{chapter:use-cases} discusses the implementation of use cases for the realized server-side clustering algorithm.

Chapter \ref{chapter:conclusions-outlook} finally evaluates the results of the thesis. It contains performance tests of the algorithm implementation and a further evaluation of the accomplishment of the objectives. Final conclusions are made and an outlook for future work is given.